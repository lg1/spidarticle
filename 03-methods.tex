\section{Materials and methods}
\label{sec:methods}

\subsection{Preparation of electrospun spidroin substrate}
\subsubsection{Materials}
	\label{sec:spidroin fabrication}

\subsubsection{Electrospining of PCL and spiroin nanofibers}
	\label{sec:electrospinning}
Both spidroin types were dissolved in hexfluorisopropanol at 15-25\% concentration. PCL solution concentration was equal to 10-15\%.	
The prepared PLC and spidroin solutions were electrospun using Nanon-01 electrospining setup(MECC CO.,LTD). 
All these solutions were loaded into 2 ml syringe and ejected through 20 gauge blunt tip needle at flow rate of 1ml/h.
The applied voltage between syringe tip and grounded collector was in the range from 5 to 10 kV. 
Nanofibers were electrospun directly onto the surface of 12mm and 22mm diameter cover glass deposited on the grounded collector.
After completing electrospinning process the spidroin fibrous substrates were immersed in 96\% ethanol for 24 hours in order to formate fibrills on the fibers(???).
Then the spidroin substrates were removed from ethanol solution, washed in PBS for 24 hours and afterwards dried.  
Non-adhesive to live cells PCL nanofibers sustrates were coated with fibronectin solution (0.16mg/ml) for accomodation of cell adhesion.
To compare adhesive properties of both spidrion mesh types (1f9 and 2e12) to fibronectin coating and PCL fibrous mesh we removed nanofibers from the half of the specimen using adhesive tape and resulting free of spidroin nanofibers half was subsequently covered with fibronectin.
     
\subsection{Cardiac  cell isolation, seeding cultivation}
Cardiac cells were isolated according to Wornigton protocol (http://www.worthingtonbiochem.com/$\newline$NCIS/default.html).
Cardiac cells were isolated from the ventricles of 1-3 day old neonatal Wistar rats. 
\subsection{Cardiac cells adhesion assay}
To perform cell adhesion analysis to the different surfaces (1f9,2e12 spidroin meshes of various fiber diameters, PCL various diameter fibers covered with fibronectin, fibronectin covered glass) we placed our substrate specimens into 24-well plate.
Then we seeded isolated rat cardiac cells at the concentration of 1.5-3*10$^{5}$ cells/cm$^{2}$.
After 5 hours cells were washed in PBS for 10 minutes by shaking on nutator. 
Afterwards we fixed cells in 4\% paraformaldehyde and labeled them with Alexa Fluor 488 Phaloidinin Conjugate for actin filaments staining and DAPI for outlining of nuclei according to protocol described in \cite{Orlova2011}. 
Images of the resulting specimens were captured with IX-71 inverted fluorescent microscope. 
Then we counted the number of alive cells on these images using ImageJ software and performed comparative analysis of adhesion properties of our substrates.     
\subsection{Scanning electron microscopy}
Before SEM imaging all specimens were covered by thin 10nm layer of gold utilizing 150R/ES sputter coater (Quorum Technologies, UK).
JEOL 6510-A was used for characterization of shape and diameter of nanofibers. Accelerating voltage was 10kV.
ImageJ software and Wolfram Mathematica were used for automated measurement of fiber diameter at each acquired photograph.
Average fiber diameter size was calculated according to the size distribution histogram.
\subsection{Fluorescent cardiospecific markers labeling and imaging}      
Immunofluorescent staining of cardiomyocytes monolayers on spidroin substrates using secondary and primary antibodies was performed according to protocol previously described\cite{Bursac2002}.
In our work we used anti-$\alpha$-actinin, anti-connexin43 primary antibodies for labeling cardiomyocytes specifically and DAPI for labeling nuclei of the cells.
\subsection{Optical mapping of excitation waves}
4-5 days monolayer cultures were optically mapped using Ca$^{2+}$ Fluo-4 AM(Invitogen, USA) as described earlier\cite{Orlova2011}.
Using 500 um width platinum wire and groundedd electrode cathodal external point stimuli was applied at various frequencies and placed at different sites of the sample. 
Excitation of initiated cardiac waves were recorded by Olympus MVX-10 Macro-View fluorescent microscope equipped with high-speed Andor EM-CCD Camera 897-U at 68 fps.
\subsubsection{Analysis of optical mapping data}
All videos were processed in ImageJ software(NIH, Maryland, USA, http://rsb.info.nih.gov/ij). 
Pseudocardigrams of spidroin covered half and fibronectin covered half were plotted to compare the frequency response to external point stimuli.
In addition to frequency comparative analysis we've made conduction velocity calculation on each half of the substrate for the subsequent comparison.



